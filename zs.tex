\documentclass[UTF8]{ctexart}
\setCJKmainfont{Sarasa Mono CL}
\usepackage{listings}
\usepackage{fontspec}
\usepackage{amsmath}
\usepackage{color}
\usepackage{geometry}
\geometry{a4paper,scale=0.8} 

\definecolor{dkgreen}{rgb}{0,0.6,0}
\definecolor{gray}{rgb}{0.5,0.5,0.5}
\definecolor{mauve}{rgb}{0.58,0,0.82}

\lstset{
  language=C++,                   % the language of the code
  basicstyle=\footnotesize,       % the size of the fonts that are used for the code
  numbers=left,                   % where to put the line-numbers
  numberstyle=\tiny\color{gray},  % the style that is used for the line-numbers
  stepnumber=1,                   % the step between two line-numbers. If it's 1, each line 
                                  % will be numbered
  numbersep=5pt,                  % how far the line-numbers are from the code
  backgroundcolor=\color{white},  % choose the background color. You must add \usepackage{color}
  showspaces=false,               % show spaces adding particular underscores
  showstringspaces=false,         % underline spaces within strings
  showtabs=false,                 % show tabs within strings adding particular underscores
  frame=single,                   % adds a frame around the code
  rulecolor=\color{black},        % if not set, the frame-color may be changed on line-breaks within not-black text (e.g. commens (green here))
  tabsize=2,                      % sets default tabsize to 2 spaces
  captionpos=b,                   % sets the caption-position to bottom
  breaklines=true,                % sets automatic line breaking
  breakatwhitespace=false,        % sets if automatic breaks should only happen at whitespace
  title=\lstname,                 % show the filename of files included with \lstinputlisting;
                                  % also try caption instead of title
  keywordstyle=\color{blue},      % keyword style
  commentstyle=\color{dkgreen},   % comment style
  stringstyle=\color{mauve},      % string literal style
  escapeinside={\%*}{*)},         % if you want to add LaTeX within your code
  morekeywords={ASTptr}            % if you want to add more keywords to the set
}

\title{LLVM影响下的编译器新生态}
\author{高一4班 - 梁韬 - GitHub@Origami404}
\begin{document}
\maketitle
\section{前言}
21世纪以来, 新的编程语言的产生速度越来越快, 包括DSL在内的全新编程语言
对于编译技术的质和量的要求也越来越高. 而反观Intel等硬件厂商, 指令集也
越发复杂, 同时编译优化技术特别是GC及JIT技术的发展对于编译器作者的水平
也提出了极高的要求. 面对这种困难, LLVM诞生了.


LLVM是一个\textbf{自由软件项目}, 是一套编译器基础设施, 基于C++.
LLVM的命名最早源于所谓\textbf{底层虚拟机(Low Level Virtual Machine)}
但随着其本身的发展及壮大, 现在的LLVM只是单纯的作为一个无意义的名称, 用于
统称LLVM项目下产生的各种工具, 如LLVM IR, lldg, LLVM C++ Standard Library
等.


本综述主要叙述LLVM对于编译器编写这一传统领域所带来的影响
及基于LLVM编写编译器前端的基本流程.

\section{LLVM对传统编译编译器编写的影响}
传统的编译器实现一般包括两大部分七大模块. 
第一个部分是\textbf{分析}.
分析部分将源代码分解为多个组成要素, 并在这些组成要素上添加语法结构.
在此部分编译器将捕获并处理全部的词法与语法错误, 收集整理与代码有关的
其他信息, 最后构造出一种\textbf{中间表示(IR)}.


第二个部分是\textbf{综合}.
在此部分编译器开始着手优化并产生目标代码. 在这一部分编译器可能使用某些
巧妙的算法进行机器无关的代码优化, 又或者使用某种依赖于目标平台或目标语言
特性的优化机理进行优化, 最后产生的便是目标代码.


在此划分之上编译器又被按功能分为七个模块, 分别是\textbf{词法分析}, 
\textbf{语法分析}, \textbf{语义分析}, \textbf{中间代码生成}, 
\textbf{机器无关代码优化}, \textbf{代码生成}, \textbf{机器相关代码生成}.
其中词法分析, 语法分析, 语义分析及中间代码生成属于分析部分; 机器无关代码优化, 
代码生成, 机器相关代码生成属于综合部分.


读者不难发现, IR作为联系前端和后端的形式, 在编译器编写中具有重要作用.
在一门语言的实现中, 除了语言本身的语法定义, 就属IR的定义最为重要. 
泛观世界上成熟的编译器工具集和VM如GCC, LLVM, JVM, 甚至是LuaVM都定义了极为
成熟和在一定程度上泛用的IR表示. 这些IR为平台的二次开发奠定了基础, 也是其能流行
于世界的重要原因. 而相比起其他编译平台, LLVM的优势在于其IR的开发一开始就
基于一个目标----多语言通用, 多范式支持, 多用途使用. 因而在快速实现编译器时
LLVM具有天然的优势, 而其完全透明, 类型完备的IR也为基于该语言的开发工具提供
了实现包括语法高亮, 语义代码补全等面给用户的功能的极大便利. 诸如Apple的Xcode,
Microsoft的VSCode的C++插件等都支持对LLVM编译出的IR进行利用而改善用户体验的功能.
这也是LLVM的一大优势.

\section{利用LLVM实现编译器前端的基本流程}
LLVM C++库对中间代码生成, 优化及跨平台机器代码生成提供了简洁易用的接口.
在一般的编写流程中, 开发人员主要编写词法与语法分析器, 即俗称的Scanner与
Parser, 而后通过在Parser得出的AST继承体系中加入调用LLVM生成IR的方法
来生成LLVM IR, 极大的简化了开发流程. 对于某些对速度要求不高的需求, 
开发人员甚至可以使用诸如Bison和flex等分析器生成器生成语法词法分析器,
从而极大地降低了编译器开发的门槛. 同时也因为LLVM IR的生成一般基于AST, 
而在AST层面往往可以对语言作出极其灵活的修改而不改变用户代码的感知, 故这种
方法也具有极高的灵活性, 甚至在某些情况下能实现超出Python的metaclass和
Java的Reflection的令人惊奇的灵活修改, 能在源代码层面实现原本许多嵌入式语言中
在宿主层面才实现的特性.


为更好的介绍LLVM C++接口, 在这里实现一个简短的带变量二元运算解析.
我们的目标是支持诸如:
\begin{lstlisting}[language = Python, numbers=left, numberstyle=\tiny]
    def x = 2.44
    def y_s = 300.45 + x
    def c = (x + y_s)*3
    c * 2 + x / y_s
\end{lstlisting}
的变量赋值与二元运算. 二元运算支持+, -, *, / 与(), 为了简便所有
量的类型均为double. 更严谨的, 此语言的BNF如下: (也是一个LL(1)文法)
\begin{equation}
    programmer \; \rightarrow \; define \; | \; expression 
\end{equation}
\begin{equation}
    define \; \rightarrow \; \textbf{def} \; Id \; \textbf{=} \; expression 
\end{equation}
\begin{equation}
    expression \; \rightarrow \; primary \; [binop]
\end{equation}
\begin{equation}
    primary \; \rightarrow \; Id \; 
                    | \; Num \; 
                    | \; \textbf{(} \; expression \; \textbf{)}
\end{equation}
\begin{equation} 
    binop \; \rightarrow \; primary \; Op \; binop
\end{equation}
由于不是重点, 这里直接给出AST以及Scanner和parser的定义:
\begin{lstlisting}[title=头文件, frame=shadowbox]
// 编译命令
// clang++ -g -O3 *.cpp `llvm-config --cxxflags --ldflags --system-libs --libs core`
#include "llvm/ADT/APFloat.h"
#include "llvm/ADT/STLExtras.h"
#include "llvm/IR/Constants.h"
#include "llvm/IR/DerivedTypes.h"
#include "llvm/IR/IRBuilder.h"
#include "llvm/IR/LLVMContext.h"
#include "llvm/IR/Module.h"
#include "llvm/IR/Type.h"
#include "llvm/IR/Verifier.h"
#include <algorithm>
#include <cctype>
#include <cstdio>
#include <cstdlib>
#include <map>
#include <memory>
#include <string>
#include <vector>
using std::unique_ptr;
using std::string;
using std::map;
using std::move;
using namespace llvm;
\end{lstlisting}
\begin{lstlisting}[title=AST, frame=shadowbox]
// 这里是AST节点的定义
namespace {
    class ExprAST {
        public:
        virtual ~ExprAST() {}
        // 用于生成IR
        virtual Value* codegen() = 0;
    };
    using ASTptr = unique_ptr<ExprAST>;
    class NumAST : public ExprAST { 
        double val;
        public: 
        NumAST(double v): val(v) {}
        virtual Value* codegen();
    };
    class VarAST : public ExprAST {
        string name;
        public:
        VarAST(const string& n): name(n) {}
        virtual Value* codegen();
    };
    class BinOpAST : public ExprAST {
        char op;
        unique_ptr<ExprAST> lhs, rhs;
        public:
        BinOpAST(char op, ASTptr LHS, ASTptr RHS) 
            : op(op), lhs(move(LHS)), rhs(move(RHS)) {}
        virtual Value* codegen();
    };
    class DefAST {
        string name;
        ASTptr val;
        public:
        DefAST(const string& name, ASTptr val) 
            : name(name), val(move(val)) {}
        virtual void codegen();
    };
}
\end{lstlisting}
下面是一些主要环境变量和辅助函数:
\begin{lstlisting}[title=环境, frame=shadowbox]
enum Token {
    tok_eof = -1,
    tok_identifier = -2,
    tok_number = -3,
    tok_def = -4,
    tok_end = -5
};

static string identifier_str;
static double num_val;

int gettok() {
    static int last_char = ' ';
    while (isspace(last_char))
        last_char = getchar();
    // identifier || def
    if (isalpha(last_char)) {
        identifier_str = last_char;
        while (isalnum(last_char = getchar()))
            identifier_str += last_char;
        // def
        if (identifier_str == "def")
            return tok_def;
        return tok_identifier;
    }
    // Number, a small bug
    if (isdigit(last_char) || last_char == '.') {
        string num_str;
        do {
            num_str += last_char;
            last_char = getchar();
        } while (isdigit(last_char) || last_char == '.');
        num_val = strtod(num_str.c_str(), nullptr);
        return tok_number;
    }
    // EOF
    if (last_char == EOF)
        return tok_eof;
    if (last_char == ';'){
        last_char = getchar();
        return tok_end;
    }
        
    // Operators 
    int this_char = last_char;
    last_char = getchar();
    return this_char;
}
// 递归下降解析器的预读字符缓冲
int cur_tok; 
int next_token() {
    return (cur_tok = next_token());
}
ASTptr Error(const char* str) {
    fprintf(stderr, "Error: %s\n", str);
    return nullptr;
} 
// 用于递归下降解析的运算符优先级表
map<char, int> op_prec;
int get_prec(char c) {
    auto prec = op_prec.find(c);
    if (prec == op_prec.end())
        return -1;
    else return prec->second;
}

ASTptr Error(const char* str) {
    fprintf(stderr, "Error: %s\n", str);  
    return nullptr;
}
\end{lstlisting} 
同时一个简单的解析器也在下方给出, 注意本代码出于演示目的忽略了很多错误处理.
\begin{lstlisting}[title=解析器, frame=shadowbox] 
ASTptr parse_expression();
unique_ptr<DefAST> parse_define();
ASTptr parse_primary();
ASTptr parse_binop(int expr_prec, ASTptr lhs);
ASTptr parse_id();
ASTptr parse_num();
ASTptr parse_paren();

/// programmer ::= define | expression
// 同时承担程序入口的责任
void parse_programmer() {
    while (true) {
        printf("ready> ");
        switch (cur_tok) {
            case tok_eof: return;
            case tok_def: parse_define()->codegen(); break;
            case tok_end: break;
            default: parse_expression()->codegen()->print(errs());
        }
        next_token();
    }
}
/// define ::= 'def' Id '=' expression
unique_ptr<DefAST> parse_define() {
    next_token(); // eat def
    auto name = identifier_str;
    next_token(); // eat '='
    next_token();
    auto val = parse_expression();
    return llvm::make_unique<DefAST>(DefAST(name, move(val)));
}
/// expr ::= primary [binop]
ASTptr parse_expression() {
    return parse_binop(0, parse_primary());
}
/// primary ::= Id | Num | paren
ASTptr parse_primary() {
    switch (cur_tok) {
        case tok_identifier:
            return parse_id();
        case tok_number: 
            return parse_num();
        case '(':
            return parse_paren();
        default:
            return Error("Unknown token in primary");
    }
}
/// paren ::= '(' expression ')'
ASTptr parse_paren() {
    next_token(); // eat (
    auto expr = parse_expression();
    if (cur_tok != ')')
        return Error("Unmatch paren");
    next_token(); // eat )
    return expr;
}
/// binop ::= primary Op binop
ASTptr parse_binop(int expr_prec, ASTptr lhs) {
    while (true) {
        int now_prec = get_prec(cur_tok);
        if (now_prec < expr_prec)
            return lhs;
            
        int binop = cur_tok;
        next_token(); 
        auto rhs = parse_primary();

        int next_prec = get_prec(cur_tok);
        if (now_prec < next_prec) 
            rhs = parse_binop(expr_prec + 1, move(rhs));
        lhs = llvm::make_unique<BinOpAST>(binop, move(lhs), move(rhs));
    }
}

ASTptr parse_num() {
    auto res = llvm::make_unique<NumAST>(num_val);
    next_token();
    return move(res);
}
 
ASTptr parse_id() {
    auto res = llvm::make_unique<VarAST>(identifier_str);
    next_token();
    return move(res);
}
\end{lstlisting}
下面就是本节讲述的重点: LLVM IR C++ API.
我们使用AST节点的codegen虚方法来递归地生成IR, 您将看到
这种生成方法是如此强大与方便, 以至于代码量还不及解析器:
\begin{lstlisting}[title=IR生成, frame=shadowbox] 
// LLVM API 文档(英文): https://llvm.org/docs/LangRef.html 
static LLVMContext my_context;
static IRBuilder<> builder(my_context);
static unique_ptr<Module> my_module;
static map<string, Value *> named_vals;
Value *LogErrorV(const char *str) {
  LogError(str);
  return nullptr;
}

Value* NumAST::codegen() {
    return ConstantFP::get(my_context, APFloat(this->val));
}
Value* VarAST::codegen() {
    Value* v = named_vals[this->name];
    if (!v)
        return LogErrorV("undefine variable name");
    return v;
}
Value* BinOpAST::codegen() {
    Value* l = this->lhs->codegen();
    Value* r = this->rhs->codegen();
    if (!l || !r)
        return nullptr;
    switch (this->op) {
        case '+':
            return builder.CreateFAdd(l, r, "addtmp");
        case '-':
            return builder.CreateFSub(l, r, "subtmp");
        case '*':
            return builder.CreateFMul(l, r, "multmp");
        case '/':
            return builder.CreateFDiv(l, r, "divtmp");
        default: return LogErrorV("unexpect binop");
    }
}
void DefAST::codegen() {
    printf("Name: %s defined\n", this->name.c_str());
    named_vals[this->name] = this->val->codegen();
} 
\end{lstlisting}
从上述代码可以看出LLVM IR C++ API的完善与简洁.在保留了
比C更贴近一点底层的表示的同时很好地维护了IR的类型信息.
完整的LLVM API可以在LLVM的网站上查询. 


使用测试样例运行程序:
\begin{lstlisting}[title=测试, language=Python]
ready> def x = 42.0;
ready> def y = 3.2*x; 
ready> x+y; 
double 1.764000e+02 
ready> 
\end{lstlisting}
可以看到LLVM会在生成IR中就为我们完成简单的常量常量折叠,
事实上在这个小例子中LLVM甚至能帮我们完成\textbf{全部}
的运算并折叠成一个常量. 注意这在LLVM的分类中
\textbf{"连优化都算不上"}, 是默认开启的. LLVM对
机器无关的代码优化能力可见一斑.


但即使是这种小例子也足以折射出LLVM IR的一个特点:
强类型. LLVM IR对于类型是极为严格的, 甚至不允许C的很多
符合常理的隐式类型转换, 这一点也在后续的优化中发挥很大的作用.

\section{总结与参考引用}
本综述阐述了基于LLVM编写编译器前端的基本流程, 为今后开展
编译技术方面研究性学习奠定了坚实的基础. 今后的研究性学习
将着手于编译器前端技术的学习, 实现一个简单的Scheme解释器
与C编译器.


本文主要参考:
LLVM Tutorial : https://llvm.org/docs/tutorial/index.html


LLVM Document : https://llvm.org/docs/LangRef.html 


Wikipedia : https://www.wikipedia.org/


<<编译原理>> 机械工业出版社, 第二版.


<<深入理解计算机系统>> 机械工业出版社, 第三版.


代码地址:



\end{document}  